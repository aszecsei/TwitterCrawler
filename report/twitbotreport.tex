% This is "sig-alternate.tex" V2.1 April 2013
% This file should be compiled with V2.5 of "sig-alternate.cls" May 2012
%
% This example file demonstrates the use of the 'sig-alternate.cls'
% V2.5 LaTeX2e document class file. It is for those submitting
% articles to ACM Conference Proceedings WHO DO NOT WISH TO
% STRICTLY ADHERE TO THE SIGS (PUBS-BOARD-ENDORSED) STYLE.
% The 'sig-alternate.cls' file will produce a similar-looking,
% albeit, 'tighter' paper resulting in, invariably, fewer pages.
%
% ----------------------------------------------------------------------------------------------------------------
% This .tex file (and associated .cls V2.5) produces:
%       1) The Permission Statement
%       2) The Conference (location) Info information
%       3) The Copyright Line with ACM data
%       4) NO page numbers
%
% as against the acm_proc_article-sp.cls file which
% DOES NOT produce 1) thru' 3) above.
%
% Using 'sig-alternate.cls' you have control, however, from within
% the source .tex file, over both the CopyrightYear
% (defaulted to 200X) and the ACM Copyright Data
% (defaulted to X-XXXXX-XX-X/XX/XX).
% e.g.
% \CopyrightYear{2007} will cause 2007 to appear in the copyright line.
% \crdata{0-12345-67-8/90/12} will cause 0-12345-67-8/90/12 to appear in the copyright line.
%
% ---------------------------------------------------------------------------------------------------------------
% This .tex source is an example which *does* use
% the .bib file (from which the .bbl file % is produced).
% REMEMBER HOWEVER: After having produced the .bbl file,
% and prior to final submission, you *NEED* to 'insert'
% your .bbl file into your source .tex file so as to provide
% ONE 'self-contained' source file.
%
% ================= IF YOU HAVE QUESTIONS =======================
% Questions regarding the SIGS styles, SIGS policies and
% procedures, Conferences etc. should be sent to
% Adrienne Griscti (griscti@acm.org)
%
% Technical questions _only_ to
% Gerald Murray (murray@hq.acm.org)
% ===============================================================
%
% For tracking purposes - this is V2.0 - May 2012

\documentclass{sig-alternate-05-2015}


\begin{document}

% Copyright
\setcopyright{acmcopyright}
%\setcopyright{acmlicensed}
%\setcopyright{rightsretained}
%\setcopyright{usgov}
%\setcopyright{usgovmixed}
%\setcopyright{cagov}
%\setcopyright{cagovmixed}

%
% --- Author Metadata here ---
%\CopyrightYear{2007} % Allows default copyright year (20XX) to be over-ridden - IF NEED BE.
%\crdata{0-12345-67-8/90/01}  % Allows default copyright data (0-89791-88-6/97/05) to be over-ridden - IF NEED BE.
% --- End of Author Metadata ---

\title{Twitter Bot Behavior: How Twitter Bots Interact With People}
\subtitle{}
%
% You need the command \numberofauthors to handle the 'placement
% and alignment' of the authors beneath the title.
%
% For aesthetic reasons, we recommend 'three authors at a time'
% i.e. three 'name/affiliation blocks' be placed beneath the title.
%
% NOTE: You are NOT restricted in how many 'rows' of
% "name/affiliations" may appear. We just ask that you restrict
% the number of 'columns' to three.
%
% Because of the available 'opening page real-estate'
% we ask you to refrain from putting more than six authors
% (two rows with three columns) beneath the article title.
% More than six makes the first-page appear very cluttered indeed.
%
% Use the \alignauthor commands to handle the names
% and affiliations for an 'aesthetic maximum' of six authors.
% Add names, affiliations, addresses for
% the seventh etc. author(s) as the argument for the
% \additionalauthors command.
% These 'additional authors' will be output/set for you
% without further effort on your part as the last section in
% the body of your article BEFORE References or any Appendices.

\numberofauthors{2} %  in this sample file, there are a *total*
% of EIGHT authors. SIX appear on the 'first-page' (for formatting
% reasons) and the remaining two appear in the \additionalauthors section.
%
\author{
% You can go ahead and credit any number of authors here,
% e.g. one 'row of three' or two rows (consisting of one row of three
% and a second row of one, two or three).
%
% The command \alignauthor (no curly braces needed) should
% precede each author name, affiliation/snail-mail address and
% e-mail address. Additionally, tag each line of
% affiliation/address with \affaddr, and tag the
% e-mail address with \email.
%
% 1st. author
\alignauthor
Alic Szecsei\\
       \affaddr{University of Iowa}\\
       \email{alic-szecsei@uiowa.edu}
% 2nd. author
\alignauthor
Willem DeJong\\
       \affaddr{University of Iowa}\\
       \email{willem-dejong@uiowa.edu}
}

\maketitle
\begin{abstract}
% ABSTRACT (up to 200 words)
Twitter bots are often cited as affecting the political process by manipulating the trending topics data; similar behavior is also cited on other social platforms, such as Facebook. We present our use of unsupervised machine learning, combined with Indiana University's \emph{BotOrNot} service, to classify Twitter users as bots based on statistical analysis of their accounts, and then examine the ways in which they interact with other users.
\end{abstract}

% We no longer use \terms command
%\terms{Theory}

\section{Introduction}
% 1. INTRODUCTION (1 page)

\subsection{Background \& Motivation}
% 1.1 Background & Motivation (1-2 paragraphs)
\emph{Social bots}, also known as \emph{sybil accounts}, are programs that automate interaction on social platforms. While some may simply be humorous or helpful accounts that don't attempt to hide their status as bots, others have more manipulative goals; they may flood a social network with spam, or attempt to more subtly influence the thoughts and behavior of the humans it interacts with.

TODO: Talk about nefarious bots in history, and attempts to diminish their influence

\subsection{Problem Statement}
% 1.2 Problem Statement (1 paragraph)
While bot-detection remains a problem, we wanted to determine how these bot accounts preferred to interact with human users.

\subsection{Proposed Approach}
% 1.3 Proposed Approach (1 paragraph)
In this paper, we use data from a bot-detection service run by Indiana University to determine whether or not users are bots. We then pull their latest tweets, as well as user data, and use the collected data in an unsupervised machine learning algorithm to cluster the users into 50 groups.

We then take the data for each cluster and analyze common behavioral patterns.

\subsection{Key Results}
% 1.4 Key Results (1-2 paragraph)
We found a general inverse trend between the \emph{BotOrNot} score for a cluster and the number of retweets made by the cluster. In addition, a similar inverse trend exists for the number of links tweeted by users.

\section{Related Work}
% RELATED WORK (1/2 page)
Lorem ipsum

\section{Proposed Approach}
% PROPOSED APPROACH (1 page)
Lorem ipsum

\section{Results \& Discussion}
% RESULTS & DISCUSSIONS (3 pages including tables and figures)
Lorem ipsum

\section{Conclusion}
% CONCLUSION (~1/4 pages)
Lorem ipsum\cite{bowman:reasoning}

%
% The following two commands are all you need in the
% initial runs of your .tex file to
% produce the bibliography for the citations in your paper.
\bibliographystyle{abbrv}
\bibliography{sigproc}  % sigproc.bib is the name of the Bibliography in this case
% You must have a proper ".bib" file
%  and remember to run:
% latex bibtex latex latex
% to resolve all references
%
% ACM needs 'a single self-contained file'!
%
%APPENDICES are optional
%\balancecolumns
% \appendix

%\balancecolumns % GM June 2007
% That's all folks!
\end{document}
